\documentclass{template/template}

\usepackage{subcaption}
\usepackage{amsmath}
\usepackage{enumitem}
\usepackage{hyperref}
\usepackage{gensymb} % balíček symbolů
\usepackage{booktabs}

\usepackage[toc,page]{appendix}
\usepackage{color} % balíček pro obarvování textů
\usepackage{xcolor}  % zapne možnost používání barev, mj. pro \definecolor
\definecolor{mygreen}{RGB}{0,150,0} % nastavení barev odkazů 
\usepackage{listings} % balíček pro formátování zdrojových kódů 
\usepackage[author=,status=draft]{fixme} % vkládání poznámek  
% dva módy (status): draft (poznámky se zobrazují v PDF) / final (poznámky se nezobrazují v PDF)
\usepackage{multirow}

\lstset { %
    language=C++,
    backgroundcolor=\color{black!5}, % set backgroundcolor
    basicstyle=\footnotesize,% basic font setting
}

\addbibresource{text.bib}
\nocite{*}

\titlecz{Tvorba univerzální šablony práce pro LaTeX} % Název práce
\titleen{LaTeX universal thesis template creation} % Anglický název práce
\author{Franta Vokřál} % Jméno autora
\institution{Nějaká organizace} % Celý název instituce
\institutiontype{příspěvková organizace} % Typ instituce
\thesistype{Bakalářská práce}  % Typ práce/dokumentu
\mentor{Ing. Jan Novák} % Jméno vedoucího práce
\mentorstatement{Ing. Jan Novák} % Jméno vedoucího práce ve čtvrtém pádě
\field{Psaní šablon} % Okruh, nebo téma

\placefooter{Brno 2021}

%\usepackage{hyperref} % balíček pro hypertextové odkazy
% \url{www.odkaz.cz}
% \href{http://www.odkaz.cz}{Text který bude jako odkaz}
% \hyperlink{label}{proklikávací_text} - odkaz na text 
% \hypertarget{label}{cíl_odkazu} - cíl odkazu 

\begin{document}

\maketitle

%\makecopyrightstatement{V~Brně}

\makethanks{Duis ante orci, molestie vitae vehicula venenatis, tincidunt ac pede. Nemo enim ipsam voluptatem quia voluptas sit aspernatur aut odit aut fugit, sed quia consequuntur magni dolores eos qui ratione voluptatem sequi nesciunt. Proin pede metus, vulputate nec, fermentum fringilla, vehicula vitae, justo. Pellentesque arcu. Cum sociis natoque penatibus et magnis dis parturient montes, nascetur ridiculus mus. Fusce aliquam vestibulum ipsum. Aenean placerat. Excepteur sint occaecat cupidatat non proident, sunt in culpa qui officia deserunt mollit anim id est laborum. Vivamus porttitor turpis ac leo. Nullam eget nisl. Maecenas libero. Nunc auctor. Nullam rhoncus aliquam metus. Integer pellentesque quam vel velit.}

\pagestyle{empty}

\section*{Anotace}
Sem patří anotace v češtině.

\subsection*{Klíčová slova}
5 a příp. více klíčových slov

\vspace{20mm}

\section*{Annotation}
Here goes english version of thesis annotation.
\fxnote[author=PŠ]{Nějak takto vypadá poznámka vytvořená přes fxnote}

\fxnote[author=PŠ]{\textcolor{mygreen}{A dokonce je lze obarvit!}}

\subsection*{Keywords}
Here goes 5 or more keywords

\newpage
\pagestyle{plain}

\tableofcontents % vysází obsah

%%% Začátek práce
\setcounter{figure}{0}
\setcounter{table}{0}
\newpage

% Uvod prace
\chapter*{Úvod}
\addcontentsline{toc}{chapter}{Úvod}
Sem přijde úvod práce.

\newpage


\chapter{Kapitola --}
Zde vidíte ukázkovou kapitolu. 
Kapitoly jsou rozděleny do jednotlivých souborů ve složce CHAPTERS pro větší přehlednost. 

\newpage

% Zaver prace
\chapter*{Závěr}
\addcontentsline{toc}{chapter}{Závěr}
Sem přijde závěr práce.

\newpage
\newpage

\appendix
\addcontentsline{toc}{chapter}{Přílohy}

% Prilohy
\chapter{Obrazové přílohy}
Do této kapitoly patří obrazové přílohy (zpravidla velké fotky/grafy/diagramy apod., které není vhodné vkládat do textu). 
Formát vložení obrazu vidíte níže (v kódu).
Stejným způsobem se dají vytvořit oddělené kapitoly pro další typy příloh.

\begin{figure}[h]
    \centering
    \includegraphics[width=0.85\textwidth]{img/ToBeRemoved/PPSB-T_BOTH.png}
    \caption{Vizualizace PPSB-T (horní strana vpravo, dolní vlevo).}
    \label{fig:PPSB-T_VISUAL}
\end{figure}

\printbibliography[title=Literatura]
\addcontentsline{toc}{chapter}{Literatura}

\listoffigures
\addcontentsline{toc}{section}{Seznam obrázků}

\listoftables
\addcontentsline{toc}{section}{Seznam tabulek}

\end{document}
